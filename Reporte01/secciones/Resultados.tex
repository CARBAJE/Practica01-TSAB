\chapter{Resultados y Discusión}

Durante la ejecución del algoritmo genético para la mejora de contraste en imágenes médicas, se realizaron múltiples corridas completas para cada función objetivo (Maximización de Entropía y Maximización de Desviación Estándar), lo que permitió evaluar la estabilidad y eficiencia del método. Los resultados se agruparon en resúmenes globales, donde se registraron indicadores clave, los cuales se encuentran dentro de las siguientes tablas:

\begin{table}[H]
\centering
\caption{Resumen Global de Corridas 01 - Imagen Médica 5R (Mano)}
\begin{tabular}{lcc}
\hline
\textbf{Indicador} & \textbf{Optimización de Entropía} & \textbf{Optimización de Desviación Estándar} \\
\hline
Mejor (Fitness) & -4.128421621722379 & -0.1953755630183538 \\
Peor (Fitness) & -4.128421621722379 & -0.1953755555673538 \\
Media & -4.128421621722379 & -0.1953755630183538 \\
Desv. Estándar & 0.0 & 7.51e-09 \\
\hline
\end{tabular}
\end{table}

\begin{table}[H]
    \centering
    \caption{Resumen Global de Corridas 02- Imagen Médica 5R (Mano)}
    \begin{tabular}{lcc}
    \hline
    \textbf{Indicador} & \textbf{Optimización de Desviación Estándar} \\
    \hline
    Mejor (Fitness) & 120.061897 \\
    Peor (Fitness) & 120.245979 \\
    Media & 120.14108900000001 \\
    Desv. Estándar & 0.07353604532132446 \\
    \hline
    \end{tabular}
\end{table}


\begin{table}[H]
    \centering
    \caption{Resumen Global de Corridas 01 - Imagen Médica 6R (Ojo)}
    \begin{tabular}{lcc}
    \hline
    \textbf{Indicador} & \textbf{Optimización de Entropía} & \textbf{Optimización de Desviación Estándar} \\
    \hline
    Mejor (Fitness) & -6.84574278505749 & -0.2244060136302129 \\
    Peor (Fitness) & -6.84574278505749 & -0.2244058045302129 \\
    Media & -6.84574278505749 & -0.2244060136302129 \\
    Desv. Estándar & 0.0 & 1.91e-05 \\
    \hline
    \end{tabular}
\end{table}

\begin{table}[H]
    \centering
    \caption{Resumen Global de Corridas 02 - Imagen Médica 6R (Ojo)}
    \begin{tabular}{lcc}
    \hline
    \textbf{Indicador} & \textbf{Optimización de Entropía} & \textbf{Optimización de Desviación Estándar} \\
    \hline
    Mejor (Fitness) & 4.7195 & 122.711517 \\
    Peor (Fitness) & 4.7211 & 122.773857 \\
    Media & 4.72044 & 122.7600892 \\
    Desv. Estándar & 0.0006066300355240354 & 0.027192434392680088 \\
    \hline
    \end{tabular}
\end{table}

Donde cada Indicador Representa lo siguiente:
\begin{itemize}
    \item \textbf{Mejor (Fitness):} Representa la solución con el valor de fitness mínimo obtenido en todas las corridas.
    \item \textbf{Peor (Fitness):} Indica la solución con el mayor valor de fitness, sirviendo como referencia de la variabilidad en la búsqueda.
    \item \textbf{Media:} Es el promedio de los valores de fitness de la mejor solución de cada corrida, ofreciendo una visión global del desempeño del algoritmo.
    \item \textbf{Desv. Estándar:} Mide la dispersión de los valores de fitness entre las corridas, reflejando la estabilidad y consistencia del proceso evolutivo.
\end{itemize}

\section{Análisis de los Resultados}

\subsection{Consistencia y Robustez}
Los resúmenes globales muestran que, a lo largo de las corridas, el algoritmo genético tiende a converger de manera consistente hacia soluciones de alta calidad. La extremadamente baja desviación estándar en los valores de fitness, especialmente para la optimización de entropía (0.0 en ambas imágenes), sugiere que el proceso evolutivo es excepcionalmente robusto y no depende en exceso de la aleatoriedad inherente a los operadores genéticos. Esto es crucial para problemas de optimización en imágenes médicas, ya que garantiza que la metodología aplicada es reproducible y confiable para aplicaciones clínicas.

\subsection{Comparación Entre Funciones Objetivo}
\subsubsection{Optimización de Entropía}
El resumen global para la optimización de entropía indica que el algoritmo fue capaz de identificar soluciones consistentes, con una desviación estándar de 0.0 en ambas imágenes. Para la Imagen 5R, se obtuvieron valores óptimos de $\alpha = 8.2005$ y $\delta \approx 0$, mientras que para la Imagen 6R, curiosamente, los valores óptimos fueron $\alpha \approx 0$ y $\delta \approx 0$. Esta diferencia significativa entre las dos imágenes sugiere que la optimización basada en entropía es altamente sensible a las características específicas de cada imagen médica.

\subsubsection{Optimización de Desviación Estándar}
Para la optimización de la desviación estándar, los resultados globales evidencian una convergencia hacia regiones con valores de fitness bajos, con una ligera variación entre corridas. Es notable que, para ambas imágenes, esta función objetivo condujo a valores de alpha cercanos al límite superior (aproximadamente 10), pero con diferentes valores de delta: cercano a 0 para la Imagen 5R y aproximadamente 0.41 para la Imagen 6R. Esto sugiere que la maximización de la desviación estándar tiende a favorecer transformaciones con pendientes pronunciadas, independientemente de la imagen específica.

\subsection{Parámetros Óptimos y Sensibilidad}
Los resultados muestran una clara diferencia en la sensibilidad de los parámetros:

\begin{itemize}
    \item \textbf{Parámetro Alpha:} Controla la pendiente de la transformación sigmoidea y, por tanto, la intensidad del contraste. La optimización de la desviación estándar consistentemente empuja este parámetro hacia su límite superior (10), mientras que la optimización de entropía muestra una mayor variabilidad dependiendo de la imagen.

    \item \textbf{Parámetro Delta:} Controla el desplazamiento del punto medio de la transformación, afectando al brillo general. Este parámetro muestra una mayor sensibilidad y especificidad según la imagen, especialmente para la optimización de la desviación estándar.
\end{itemize}

\subsection{Transformación Sigmoidea}
La mejora de contraste se logra mediante una transformación sigmoidea definida por la función:

\begin{equation}
sigmoid(x) = \frac{1}{1 + exp(-alpha \cdot (x - delta))}
\end{equation}

Donde los parámetros optimizados por el algoritmo genético son:
\begin{itemize}
    \item \textbf{alpha:} Controla la pendiente de la transformación (intensidad del contraste)
    \item \textbf{delta:} Controla el desplazamiento del punto medio (ajuste de brillo)
\end{itemize}

\section{Discusión de Resultados}

\begin{itemize}
    \item \textbf{Eficacia del Algoritmo:} Los indicadores globales extraídos de los CSV demuestran que el algoritmo genético es capaz de encontrar soluciones óptimas para la mejora de contraste en imágenes médicas, con una consistencia notable especialmente para la optimización de entropía.

    \item \textbf{Especificidad de la Imagen:} La significativa diferencia en los parámetros óptimos entre las imágenes 5R y 6R, particularmente para la optimización de entropía, indica que la mejora de contraste óptima es altamente dependiente de las características específicas de cada imagen. Esto respalda la necesidad de técnicas de mejora adaptativas e individualizadas en lugar de parámetros fijos.

    \item \textbf{Selección de Métricas:} La elección de la función objetivo impacta significativamente en los resultados de la mejora. La optimización de entropía parece estar mejor adaptada para preservar el contenido informativo, mientras que la optimización de la desviación estándar se centra en maximizar el contraste visual.

    \item \textbf{Equilibrio entre Exploración y Explotación:} La aplicación de operadores de selección por torneo, cruzamiento SBX y mutación polinomial, junto con una población de 100 individuos y 50 generaciones, ha demostrado ser efectiva para mantener un equilibrio entre la exploración del espacio de búsqueda y la explotación de las regiones prometedoras.

    \item \textbf{Potencial para Aplicaciones Clínicas:} La consistencia y robustez mostradas por los resultados sugieren que este enfoque puede ser viable para aplicaciones clínicas donde la reproducibilidad y la confiabilidad son cruciales para el diagnóstico médico.
\end{itemize}

\subsection{Visualización de Resultados}
Para complementar el análisis numérico, se generaron representaciones visuales de las imágenes originales y mejoradas, permitiendo una evaluación cualitativa de los efectos de la optimización.

\begin{figure}[H]
    \centering
    \includegraphics[width=\textwidth]{../Medical_Image-Contrast_Enhancement_using_Genetic_Algorithms/outputs/Original_Medica5R_Entropy/surface_3d_Original_Medica5R_Entropy.png}
    \caption{Comparación entre la imagen médica 5R original y su version mejorada mediante optimización de entropía}
    \label{fig:comparacion_5R_Entropia}
\end{figure}

\begin{figure}[H]
    \centering
    \includegraphics[width=\textwidth]{../Medical_Image-Contrast_Enhancement_using_Genetic_Algorithms/outputs/Original_Medica5R_Std_Deviation/surface_3d_Original_Medica5R_Std_Deviation.png}
    \caption{Comparación entre la imagen médica 5R original y su version mejoradas mediante optimización de desviación estándar}
    \label{fig:comparacion_5R_std}
\end{figure}

\begin{figure}[H]
    \centering
    \includegraphics[width=\textwidth]{../Medical_Image-Contrast_Enhancement_using_Genetic_Algorithms/outputs/Original_Medica6R_Entropy/surface_3d_Original_Medica6R_Entropy.png}
    \caption{Comparación entre la imagen médica 6R original y su versiones mejorada mediante optimización de entropía}
    \label{fig:comparacion_6R_Entropia}
\end{figure}

\begin{figure}[H]
    \centering
    \includegraphics[width=\textwidth]{../Medical_Image-Contrast_Enhancement_using_Genetic_Algorithms/outputs/Original_Medica6R_Std_Deviation/surface_3d_Original_Medica6R_Std_Deviation.png}
    \caption{Comparación entre la imagen médica 6R original y su versiones mejorada mediante optimización de entropía}
    \label{fig:comparacion_6R_Entropia}
\end{figure}
\newpage
\section{Pensamientos finales}

El algoritmo genético implementado ha demostrado ser eficaz para la optimización de parámetros de transformación sigmoidea en la mejora de contraste de imágenes médicas. Los resultados muestran que:

\begin{itemize}
    \item La optimización de entropía y la optimización de desviación estándar representan diferentes filosofías de mejora: la primera preserva el contenido informativo mientras que la segunda maximiza el contraste visible.

    \item Las diferencias significativas en los parámetros óptimos entre imágenes enfatizan la necesidad de técnicas de mejora adaptativas y específicas para cada imagen.

    \item La consistencia de convergencia a través de múltiples corridas indica la fiabilidad del algoritmo genético para encontrar soluciones óptimas.

    \item Un sistema ideal de mejora de imágenes médicas podría beneficiarse del uso de múltiples funciones objetivo o una métrica compuesta que equilibre la preservación de la información con la mejora del contraste.
\end{itemize}

Estos hallazgos sugieren que la aplicación de algoritmos genéticos para la optimización de parámetros de transformación en imágenes médicas proporciona un enfoque robusto y adaptable para mejorar la calidad de imagen, potencialmente conduciendo a diagnósticos más precisos en entornos clínicos.