\chapter{Introducción}
El análisis de imágenes médicas desempeña un papel fundamental en el diagnóstico, la planificación del tratamiento y el seguimiento de diversas condiciones médicas. Dentro de este campo, el mejoramiento del contraste se erige como una etapa crucial para optimizar la calidad visual de las imágenes, facilitando la identificación de anomalías y la extracción de características relevantes. Las imágenes médicas a menudo presentan desafíos inherentes, como bajo contraste, ruido, pérdida de señal y estructuras anatómicas complejas, lo que puede dificultar un análisis y diagnóstico precisos. En este contexto, las exploraciones por tomografía computarizada (TC) con contraste mejorado se han consolidado como una herramienta esencial en el diagnóstico médico, permitiendo una representación más exacta de las estructuras internas del cuerpo y una detección superior de lesiones. La administración de agentes de contraste está específicamente diseñada para ``iluminar'' ciertas estructuras o áreas dentro del cuerpo humano, lo que conduce a una identificación más precisa de posibles problemas de salud.

La necesidad de técnicas efectivas de mejora del contraste en imágenes médicas es evidente debido a las dificultades que plantea el bajo contraste y la importancia de un diagnóstico certero. La investigación en esta área es, por lo tanto, de gran relevancia e impacto, ya que aborda un desafío fundamental que afecta directamente la capacidad de los profesionales médicos para realizar diagnósticos precisos.

En los últimos años, se ha observado una creciente aplicación de la inteligencia computacional, como los algoritmos genéticos, en el ámbito del mejoramiento de imágenes médicas. Si bien existen técnicas tradicionales de mejora del contraste, que incluyen la modificación del histograma y la corrección gamma, así como técnicas avanzadas como CLAHE (Ecualización de Histograma Adaptativa con Límite de Contraste), los algoritmos genéticos (AGs) ofrecen un enfoque alternativo para abordar la complejidad de este problema. Los AGs son técnicas de optimización metaheurísticas inspiradas en el proceso de selección natural, empleadas para encontrar soluciones óptimas a problemas complejos. Estos algoritmos resultan particularmente efectivos en problemas donde la función objetivo es discontinua, no diferenciable, estocástica o altamente no lineal, características que pueden presentarse en las tareas de procesamiento de imágenes. Los AGs operan sobre una población de soluciones candidatas que evolucionan a lo largo de generaciones mediante la aplicación de operadores de selección, cruce y mutación. La tendencia hacia el uso de la inteligencia computacional para la mejora de imágenes sugiere que los métodos tradicionales pueden tener limitaciones o que las técnicas más recientes ofrecen ventajas en escenarios específicos. Los algoritmos genéticos, con su capacidad para manejar problemas de optimización complejos, se presentan como una técnica avanzada viable.

El presente reporte se centra en el desarrollo de un algoritmo genéticopara el mejoramiento de contraste en imágenes médicas utilizando transformación sigmoide. La investigación propuesta realiza una comparación entre la maximización de la entropía de la imagen y la maximización de la desviación estándar de la intensidad de los píxeles como funciones objetivo dentro del proceso de optimización. El estudio utilizará una función sigmoide para la transformación de la intensidad de la imagen, y el algoritmo genético optimizará los parámetros de esta transformación. El proceso de optimización estará guiado por dos funciones objetivo distintas: la maximización de la entropía de la imagen y la maximización de la desviación estándar de la intensidad de los píxeles.
