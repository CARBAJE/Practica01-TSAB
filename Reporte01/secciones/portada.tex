% portada
\begin{titlepage}
    \begin{center}
        \vspace*{1cm}

        \begin{tabular}{c@{\hspace{2cm}}c}
            \includegraphics[width=0.25\textwidth]{\logoInstitucion} &
            \includegraphics[width=0.4\textwidth]{\logoUniversidad}
        \end{tabular}

        \vspace{1cm}

        \textbf{\LARGE \nombreInstituto} \\
        \textbf{\Large \facultad} \\
        \vspace{0.5cm}
        \textbf{\large Materia: \materia} \\
        \textbf{\large Grupo: \grupo} \\
        \vspace{0.5cm}
        \textbf{\large Profesor: \profesora} \\
        \textbf{\large Periodo: \periodo} \\

        \vspace{0.75cm}

        \textbf{\LARGE Practica 01} \\
        \vspace{0.5cm}
        \textbf{\Large \textit{Maximizar Contraste en Imagenes Medicas.}} \\

        \vspace{0.3cm}

        \textbf{\large Realizado por:} \\
        \textbf{\large \alumnoA \\ \alumnoB \\ \alumnoC \\ \alumnoD}

        % Bloque de Resumen y Abstract
        \begin{minipage}{0.8\textwidth}
            \textbf{Abstract:}\\[0.3cm]
            Medical image processing is key for diagnosis, where contrast enhancement is essential for the correct identification of structures. In this practice, a bio-inspired algorithm is implemented to optimize metrics such as standard deviation (STD) or entropy to improve visual quality without distorting the original information. Using techniques based on swarm intelligence and natural evolution, the method dynamically adjusts the contrast enhancement parameters. The results demonstrate better visual perception and greater discrimination of key anatomical structures, facilitating medical interpretation.

        \end{minipage}

        \begin{minipage}{0.8\textwidth}
            \textbf{Resumen:}\\[0.3cm]
            Las imágenes médicas suelen presentar bajo contraste, lo que dificulta su análisis y diagnóstico. Para abordar este problema, se propone un algoritmo bioinspirado que optimiza métricas como la desviación estándar y la entropía, permitiendo mejorar la calidad visual sin comprometer la información original. Este enfoque adaptativo maximiza el detalle de las estructuras relevantes, facilitando su interpretación y apoyando la toma de decisiones clínicas.
        \end{minipage}
        
        \vspace{0.3cm}

        \textbf{\large Fecha: \today}

    \end{center}
\end{titlepage}
